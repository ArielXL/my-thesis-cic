\chapter{Conclusiones}

En primer lugar, se estableció un marco de referencia para determinar el estadio del cáncer de mama (inicial o avanzado) en imágenes mamográficas a partir del análisis y la anotación del conjunto de datos público CBIS-DDSM. Como resultado de esta etapa, se definieron criterios claros para clasificar las lesiones según su grado de avance; por ejemplo, las lesiones avanzadas se asociaron típicamente a características radiológicas más agresivas, mientras que las lesiones en estadio inicial presentaron tamaños menores y contornos más circunscritos. Esta caracterización permitió fundamentar el condicionamiento del modelo generativo en el estadio tumoral, a la vez que garantizó una adecuada representación en los datos de la distinción entre casos tempranos y avanzados.

En segundo lugar, se diseñó e implementó un modelo de red generativa adversarial condicional (cGAN) capaz de generar imágenes mamográficas sintéticas de lesiones usando como condición de entrada el estadio del tumor. El modelo aprendió las características morfológicas distintivas de los estadios inicial y avanzado, logrando producir proyecciones plausibles de cómo podría verse una lesión en un estadio más avanzado en comparación con su etapa temprana. Estas imágenes sintéticas mantienen rasgos realistas, imitando la apariencia de las mamografías originales en cuanto a la distribución de densidad, el contraste de los tejidos y la definición de los bordes de las lesiones. De esta manera, el modelo consigue diferenciar adecuadamente los tumores incipientes de aquellos más desarrollados en las imágenes generadas. Sin embargo, es importante resaltar que el modelo no proporciona una progresión temporal real de las lesiones: los pares de imágenes generados para distintos estadios constituyen pseudopares conceptuales (es decir, una lesión inicial hipotética y su contraparte avanzada igualmente hipotética) y no corresponden a la evolución de un mismo tumor a lo largo del tiempo. Por lo tanto, no es posible inferir un intervalo de tiempo específico de progresión tumoral a partir de dichas proyecciones sintéticas.

En tercer lugar, se evaluó la calidad, el realismo y la plausibilidad diagnóstica de las imágenes generadas mediante una validación llevada a cabo por radiólogos expertos. Los especialistas calificaron las imágenes sintéticas en comparación con mamografías reales, e indicaron que el modelo generó lesiones con un alto grado de realismo visual. En general, las características de las lesiones simuladas (forma, bordes, patrones de densidad) fueron consideradas verosímiles y coherentes con los estadios especificados. Esta validación cualitativa por expertos confirmó que las imágenes producidas son plausibles desde el punto de vista clínico, lo que respalda la utilidad del modelo para propósitos como la simulación de casos y la ampliación de conjuntos de datos en el entrenamiento de sistemas de diagnóstico asistido.

Adicionalmente, se constató que la calidad de las imágenes generadas dependió significativamente del proceso de preprocesamiento y de la anotación de los datos. Un correcto aislamiento de la región de interés en la mamografía y una adecuada normalización de las intensidades fueron aspectos cruciales para entrenar eficazmente el modelo cGAN y obtener resultados realistas. Por otra parte, aunque la mayoría de las imágenes sintéticas resultaron convincentes, los radiólogos pudieron identificar en algunos casos ciertos artefactos propios de la generación artificial (tales como texturas repetitivas o patrones de ruido). Este hallazgo resalta la importancia de complementar la validación cualitativa experta con métricas cuantitativas de similitud a fin de evaluar de manera integral la fidelidad de las imágenes generadas.

\section{Aportes científicos}

En síntesis, los resultados alcanzados se traducen en varios aportes científicos importantes:

\begin{itemize}
    \item Marco de referencia para estadificación en mamografías: Se generó un conjunto de datos cuidadosamente anotado y categorizado según el estadío del tumor (inicial vs. avanzado) a partir de CBIS-DDSM, proporcionando una base estandarizada para estudios sobre cáncer de mama en imágenes. Este marco de datos anotados facilita el análisis comparativo de lesiones según su grado de avance y podrá ser aprovechado por futuras investigaciones en detección y clasificación de patologías mamarias.
    \item Modelo generativo condicional de lesiones mamarias: Se diseñó una arquitectura cGAN original capaz de aprender las diferencias morfológicas entre lesiones de distinto estadío y de generar imágenes sintéticas realistas condicionadas a dicha información. Los experimentos demostraron que el modelo produce imágenes de calidad comparable a la de otros enfoques generativos de última generación, validando la efectividad de la arquitectura propuesta para data synthesis en el dominio mamográfico.
    \item Validación clínica de imágenes sintéticas: Se implementó una estrategia de evaluación inédita en este contexto, involucrando a radiólogos expertos en la validación de las imágenes generadas. Esta participación experta confirmó el alto realismo de las mamografías sintéticas y sentó un precedente para la incorporación del criterio clínico en la validación de modelos generativos. Los hallazgos demuestran que la percepción humana especializada puede integrarse como medida de calidad, complementando las métricas cuantitativas tradicionales y aportando una perspectiva clínica a la evaluación de modelos de Deep Learning.
\end{itemize}

\section{Aplicaciones}

La metodología y los resultados obtenidos abren diversas posibilidades de aplicación:

\begin{itemize}
    \item Aumento de datos para algoritmos de detección: Las imágenes sintéticas generadas pueden incorporarse como datos de entrenamiento adicionales para mejorar los sistemas de diagnóstico asistido por computadora en mamografía. El uso de \textit{data augmentation} mediante GAN ayuda a paliar la escasez de ejemplos de ciertos tipos de lesiones y el desbalance de clases en los datos. De hecho, se ha reportado que la inclusión de imágenes mamográficas sintéticas en el entrenamiento de clasificadores puede incrementar significativamente su desempeño.
    \item Formación y entrenamiento de radiólogos: Un banco de mamografías sintéticas etiquetadas por estadío tumoral puede servir como material didáctico en la capacitación de especialistas. Esto permitiría exponer a los radiólogos en formación a un espectro más amplio de presentaciones de cáncer de mama (particularmente de estadíos avanzados poco frecuentes) sin comprometer la privacidad de pacientes reales. Adicionalmente, las imágenes generadas podrían emplearse en simuladores y cursos de educación médica continua para afinar las habilidades de detección temprana.
    \item Compartición segura de datos médicos: Al ser sintéticas, las imágenes producidas carecen de identificadores asociados a pacientes reales, lo que facilita su distribución y uso compartido en la comunidad científica sin incurrir en problemas éticos o legales de privacidad. Esto abre la puerta a la creación de repositorios abiertos de mamografías sintéticas que representen diversas patologías y estadíos, impulsando la colaboración y replicabilidad en investigación.
    \item Simulación de progresión tumoral: Gracias a la capacidad del modelo condicional, es posible generar secuencias de imágenes que representen hipotéticamente la evolución de una lesión desde un estadío inicial a uno avanzado. Esta funcionalidad podría incorporarse en herramientas clínicas para ilustrar escenarios de progresión del cáncer de mama, ayudando tanto en la comprensión por parte de pacientes (educación sanitaria) como en la planificación de estrategias de tratamiento al visualizar posibles futuros estados de la lesión bajo distintas circunstancias.
\end{itemize}

\section{Trabajo a futuro}

Derivado de las limitaciones y hallazgos, se identifican varias direcciones para trabajo futuro:

\begin{itemize}
    \item Ampliación y diversificación de datos: Es prioritario incorporar más datos de entrenamiento para el modelo. En particular, integrar otras bases de datos de mamografías o recabar nuevos conjuntos locales podría aumentar la diversidad de casos. Asimismo, obtener anotaciones más granulares del estadío tumoral brindaría un marco más preciso para entrenar el modelo en escenarios clínicos reales.
    \item Mejora de la resolución y realismo de las imágenes: Explorar arquitecturas generativas más avanzadas que permitan generar mamografías sintéticas a mayor resolución. Alternativas como GAN progresivas de alta resolución, modelos StyleGAN o enfoques recientes basados en difusores podrían elevar la fidelidad de las imágenes, reproduciendo detalles sutiles y texturas finas. Un modelo de mayor resolución tendría el potencial de generar imágenes con calidad casi diagnóstica, incluyendo microcalcificaciones y otras señales tempranas que actualmente quedan fuera del alcance a 256×256 píxeles.
    \item Estadificación continua o multiclase: Extender la capacidad condicional del modelo más allá de la etiqueta binaria. Una dirección sería entrenar el cGAN para múltiples categorías de estadío, o incluso abordar el problema como una regresión continua donde el grado de avance tumoral sea un parámetro real. De este modo, el modelo podría generar imágenes correspondientes a estadíos intermedios (no solo casos iniciales/extremos), permitiendo simular de forma más realista la transición gradual entre etapas de la enfermedad.
    \item Validación clínica a mayor escala: Para fortalecer la evidencia de la utilidad del modelo, se propone realizar estudios con un número más amplio de radiólogos y casos. Por ejemplo, organizar pruebas en las que docenas de especialistas intenten distinguir imágenes reales de sintéticas, recopilando estadísticas robustas de sensibilidad/especificidad de detección de imágenes generadas. Complementariamente, incorporar medidas cuantitativas en la evaluación proporcionaría criterios objetivos de comparación con otras técnicas y con imágenes reales. Una validación más extensa ayudará a determinar en qué grado las imágenes sintéticas pueden integrarse de manera confiable al flujo de trabajo clínico.
    \item Integración en sistemas y flujo clínico: Una aplicación a futuro será incorporar el modelo generativo en sistemas de detección asistida. Por un lado, las imágenes sintéticas pueden emplearse para aumentar y balancear conjuntos de entrenamiento de algoritmos de detección de cáncer de mama, potenciando su rendimiento. Por otro lado, podría desarrollarse una herramienta donde, dada una mamografía inicial, el sistema genere una proyección simulada de la misma lesión en un estadío más avanzado, ofreciéndola como información complementaria al radiólogo. Integrar estas capacidades en estaciones de trabajo radiológicas podría enriquecer la toma de decisiones diagnósticas y de seguimiento.
    \item Extensión a otras modalidades de imagen médica: Finalmente, sería interesante evaluar la generalización del enfoque a imágenes de distinto tipo. Por ejemplo, adaptar el modelo para generar mamografías 3D de tomosíntesis o incluso imágenes histopatológicas digitales de tejido mamario podría revelar si la aproximación condicional mantiene su eficacia en otros dominios. Estos escenarios ampliarían el impacto del trabajo, aplicando la síntesis de imágenes condicionada no solo en radiografías mamarias 2D, sino también en volumetrías complejas o en láminas de biopsia, contribuyendo a un espectro más amplio de herramientas en la lucha contra el cáncer de mama.
\end{itemize}