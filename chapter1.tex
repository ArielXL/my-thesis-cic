\chapter{Introducción}

\section{Contexto global del cáncer de mama}

El cáncer de mama representa uno de los principales retos de salud pública a nivel mundial. Es el tipo de cáncer más frecuente entre las mujeres en prácticamente todos los países, con alrededor de una de cada cuatro nuevos casos de cáncer en mujeres atribuidos a esta enfermedad \cite{Ventura2024}. Según datos de la Organización Mundial de la Salud (OMS), tan solo en 2020 se diagnosticaron aproximadamente 2.3 millones de casos nuevos de cáncer de mama en el mundo, y para 2022 se estimaron cerca de 670,000 fallecimientos por esta causa \cite{oms2024}. Estas cifras posicionan al cáncer mamario no solo como la neoplasia maligna de mayor incidencia en la mujer, sino también como una de las principales causas de mortalidad por cáncer en la población femenina a nivel global \cite{Ventura2024}. A pesar de los avances en concientización y detección temprana, la carga mundial de esta enfermedad continúa en aumento, especialmente en países de ingresos bajos y medios donde el acceso a programas de tamizaje y tratamientos oportunos es limitado. En contraste, las naciones con sistemas de salud robustos han logrado mejorar la sobrevida mediante la detección temprana y tratamientos cada vez más eficaces, lo que evidencia la importancia crítica de las intervenciones tempranas para reducir la mortalidad asociada.

En México, el cáncer de mama se ha convertido igualmente en el cáncer más común entre las mujeres y en la primera causa de muerte por neoplasias malignas en la población femenina \cite{valadez2023}. Cada año se diagnostican alrededor de 25,000 nuevos casos de cáncer mamario a nivel nacional. Las estadísticas recientes son preocupantes: en 2023 se registraron aproximadamente 8,034 defunciones de mujeres por tumor maligno de mama en el país, lo que equivale a un promedio de 21 mexicanas fallecidas por esta enfermedad cada día \cite{valadez2023}. Estas cifras representan cerca del 9\% de todas las muertes por cáncer en adultos en México, reflejando la magnitud del problema.

La Ciudad de México no es ajena a esta realidad, de hecho, es una de las entidades con mayor afectación. Tan solo en 2022, la capital del país reportó 909 muertes de mujeres por cáncer de mama, consolidándola como la segunda entidad federativa con el número más alto de defunciones por esta causa, solamente por debajo del Estado de México \cite{valadez2023}. En la última década se ha observado en la Ciudad de México una tendencia ascendente en la mortalidad por cáncer mamario, un aumento acumulado de alrededor de 174 fallecimientos anuales adicionales en el período entre 2013 y 2023, atribuible en parte al diagnóstico tardío y a interrupciones en los servicios de salud, como las ocurridas durante la pandemia de COVID-19 \cite{valadez2023}. Estas estadísticas locales subrayan la urgencia de mejorar las estrategias de detección y atención en el entorno mexicano.

\begin{figure}[h]
    \includegraphics[width=10cm]{Images/mamografia.jpg}
    \centering
    \caption{Análisis de mamografía}
\end{figure}

\section{Planteamiento del problema}

A pesar de ser la herramienta de referencia para la detección temprana, la mastografía presenta limitaciones importantes para prever la progresión de los tumores de mama una vez detectados. En la práctica clínica actual, el seguimiento de una lesión sospechosa se realiza mediante estudios seriados, el radiólogo compara la mastografía inicial con las sucesivas para identificar cambios en el tamaño o las características del tumor. Este enfoque reactivo implica que no existe un método rutinario para predecir de forma visual cómo evolucionará un tumor entre una mastografía y la siguiente. La incapacidad de anticipar la progresión puede llevar a demoras en la toma de decisiones; por ejemplo, entre la detección de una lesión y la confirmación de su crecimiento pueden transcurrir meses valiosos. Idealmente, contar con una proyección de la apariencia futura del tumor podría apoyar la planificación del tratamiento y la frecuencia de vigilancia, alertando tempranamente sobre posibles crecimientos agresivos.

Las limitaciones de la mastografía para pronóstico se deben tanto a factores biológicos como técnicos. Por un lado, el crecimiento tumoral varía significativamente entre pacientes, y factores como la biología molecular del tumor influyen en su velocidad de avance. Por otro lado, la mastografía es una imagen bidimensional de un proceso tridimensional, sujeta a variaciones en la posición, compresión mamaria y calidad de la imagen entre estudios, lo que dificulta extrapolar cambios futuros. En la actualidad, el pronóstico de la evolución se basa más en tablas estadísticas de riesgo y en la experiencia clínica que en evidencia visual concreta.

En años recientes ha surgido interés en aplicar métodos de inteligencia artificial para abordar este vacío. En particular, modelos generativos han sido explorados con la intención de simular la progresión de lesiones mamarias en imágenes, como una forma de predecir su apariencia en el tiempo \cite{jubran2021}. Estudios iniciales han sugerido que la simulación de la evolución de un hallazgo sospechoso en una mastografía podría contribuir a la detección más oportuna y a una mejor comprensión del comportamiento tumoral \cite{jubran2021}. Sin embargo, dichos enfoques se encuentran en etapas tempranas de desarrollo y aún enfrentan retos considerables, como la necesidad de conjuntos de datos de seguimiento (imágenes de la misma paciente en distintos momentos) y la garantía de realismo y fidelidad clínica en las imágenes generadas. En este contexto, se plantea la siguiente pregunta de investigación: ¿es posible desarrollar un modelo basado en redes generativas adversariales que pronostique la apariencia futura de un tumor de mama en mastografías, dada la imagen inicial, de manera realista y útil para apoyar la toma de decisiones clínicas? Resolver este problema implicaría dotar a la mastografía de una capacidad predictiva de la que actualmente carece, abriendo la puerta a un seguimiento más proactivo del cáncer de mama.

\section{Justificación y motivación}

La relevancia de este trabajo de investigación se sustenta en la gran carga que representa el cáncer de mama para la salud de las mujeres, tanto a nivel global como en México, y en las limitaciones de las herramientas actuales para mejorar ese panorama. Dado que el cáncer de mama es la principal causa de muerte oncológica femenina en este país \cite{valadez2023}, cualquier avance que contribuya a una detección más temprana o a un mejor entendimiento de la progresión tumoral puede traducirse en vidas salvadas.

% Justificación clínica
En México, muchas mujeres aún son diagnosticadas en etapas avanzadas de la enfermedad, lo cual se refleja en las elevadas tasas de mortalidad. Diversos factores contribuyen a esta situación, entre ellos la baja cobertura de los programas de tamizaje mamográfico. De acuerdo con encuestas nacionales de salud, solo alrededor del 27\% de las mujeres que cumplen con el perfil de edad para tamizaje (40-69 años) se realizan una mastografía de manera periódica \cite{valadez2023}, muy por debajo del estándar internacional recomendado. Esta deficiencia en la detección oportuna implica que un gran número de tumores se descubren tarde, cuando ya son evidentes clínicamente. Contar con una herramienta que, tras detectar un tumor en una mastografía inicial, permita pronosticar su crecimiento o cambios morfológicos, podría ayudar a los médicos a estratificar riesgo y priorizar recursos: por ejemplo, pacientes con un tumor proyectado a crecer rápidamente podrían ser candidatas a evaluaciones más frecuentes o a tratamientos más agresivos desde el inicio.

% Justificación científica y tecnológica
Desde el punto de vista de la investigación en imágenes médicas e inteligencia artificial, el pronóstico de la apariencia tumoral es un tema de frontera. Aplicar redes generativas adversariales (GAN) condicionales en el ámbito de la mastografía ofrece la posibilidad de sintetizar imágenes futuras que emulen la progresión de la enfermedad. Esto no solo tendría impacto clínico potencial, sino que también enriquecería el conocimiento técnico sobre cómo integrar información temporal en modelos de aprendizaje profundo para imágenes estáticas. Una GAN capaz de aprender las transformaciones entre una imagen inicial y una futura podría revelar patrones subyacentes del crecimiento tumoral y servir como base para desarrollar sistemas de apoyo al diagnóstico más sofisticados.

\section{Objetivos}

\subsection{Objetivo general}

Proponer una metodología basada en redes generativas adversariales condicionales de arquitectura compacta, capaz de pronosticar la apariencia futura de tumores de seno en imágenes de mastografía a partir de un estudio inicial. Este modelo de inteligencia artificial deberá generar imágenes sintéticas realistas que representen la posible evolución de una lesión mamaria en el intervalo de tiempo entre mastografías, con el fin de proporcionar a los radiólogos una herramienta de apoyo en la toma de decisiones respecto al seguimiento y tratamiento.

\subsection{Objetivos específicos}

\begin{enumerate}
    \item Establecer un marco de referencia para el estadío (inicial o avanzado) del cáncer de mama en mamografías, a partir de la anotación y análisis del conjunto de datos público CBIS - DDSM.
    
    \item Diseñar un modelo de red generativa adversarial condicional (cGAN) para generar proyecciones morfológicas plausibles de lesiones mamarias en función del estadío del tumor.

    \item Evaluar la calidad, realismo y plausibilidad de las imágenes generadas por el modelo, mediante la validación experta por radiólogos.
\end{enumerate}

\section{Hipótesis}

La hipótesis de esta investigación es que una red generativa adversarial condicional (cGAN) entrenada con mastografías debidamente anotadas puede generar imágenes sintéticas realistas que simulen la evolución en la apariencia de tumores mamarios. En otras palabras, se postula que a partir de una imagen inicial de mastografía, el modelo será capaz de predecir cómo podría verse la lesión en un momento futuro, reflejando cambios plausibles en tamaño, forma y otras características relevantes del tumor. Esta hipótesis está alineada con los objetivos planteados, ya que, de confirmarse, se demostraría la viabilidad de pronosticar la progresión tumoral mediante técnicas de inteligencia artificial, aportando una herramienta novedosa para apoyar el diagnóstico y seguimiento del cáncer de mama.

\section{Solución propuesta}

Para abordar la problemática descrita y dar cumplimiento a los objetivos de la investigación, se propone el desarrollo de una red generativa adversarial condicional (cGAN) de arquitectura compacta. Esta red será entrenada con la base de datos \textit{Curated Breast Imaging Subset of Digital Database
for Screening Mammography} (CBIS-DDSM) \cite{Lee2017}, la cual provee un amplio conjunto de mastografías con sus lesiones anotadas por expertos . El propósito es que el modelo generado sea capaz de producir imágenes de mastografía sintéticas pronosticadas a partir de una imagen inicial, simulando cómo podría manifestarse el tumor en estudios posteriores de la misma paciente.

La arquitectura condicional propuesta opera bajo el esquema de traducción de imágenes (image-to-image translation) asistida por aprendizaje adversarial. El generador de la cGAN toma como entrada la mastografía inicial (que contiene un tumor con un estado conocido) y genera una nueva imagen que refleja un posible estado futuro de la lesión. Simultáneamente, un discriminador convolucional evalúa la autenticidad de la mastografía sintética comparándola con imágenes reales provenientes de casos clínicos similares, lo que obliga al generador a refinar sus predicciones para que resulten indistinguibles de las mastografías reales. A través de este entrenamiento adversarial condicional, el modelo aprende a producir pronósticos visuales realistas, manteniendo la coherencia anatómica con respecto a la mama y la lesión originales.

La solución propuesta enfatiza dos criterios fundamentales: la plausibilidad clínica de los resultados y la eficiencia computacional del modelo. Por un lado, las imágenes generadas deben presentar características y patrones consistentes con hallazgos médicos reales, de forma que un especialista las consideraría verosímiles. Para fomentar esta plausibilidad clínica, se aprovechan las detalladas anotaciones proporcionadas en CBIS-DDSM durante el entrenamiento, de manera que el modelo disponga de información explícita sobre la ubicación y extensión de las lesiones al aprender la progresión simulada. Por otro lado, la cGAN se diseña con una arquitectura compacta optimizada, reduciendo el número de parámetros y la complejidad computacional, lo cual permite entrenar el modelo con recursos de hardware moderados y generar pronósticos de manera ágil, cumpliendo así con el criterio de eficiencia computacional. Este equilibrio entre realismo clínico y bajos requerimientos de cálculo es clave para la viabilidad de la solución propuesta en entornos clínicos reales.

\section{Limitaciones}

A pesar de los importantes aportes, la presente investigación presenta ciertas limitaciones que deben considerarse:

\begin{itemize}
    \item Dimensión y calidad del conjunto de datos: Si bien CBIS-DDSM es un referente público en mamografía, contiene un número relativamente reducido de casos y sus imágenes provienen de mamografías analógicas digitalizadas (con calidad y resolución inferiores a las mamografías digitales modernas). Esta limitación en volumen y calidad de datos puede restringir la capacidad del modelo para generalizar a imágenes de otras fuentes o a detectar variaciones morfológicas muy infrecuentes.
    \item Resolución de las imágenes generadas: Por restricciones computacionales, las imágenes sintéticas se generaron a una resolución máxima del orden de $256 \times 256$ píxeles. Esta resolución limitada implica que ciertos detalles (por ejemplo, microcalcificaciones) no puedan reproducirse con fidelidad, y se ha señalado que trabajar con tamaños de imagen reducidos puede impedir capturar toda la variabilidad visual, además de no garantizar la eliminación completa del problema de mode collapse en el entrenamiento. Es posible que imágenes de mayor resolución requieran arquitecturas más complejas o entrenamiento adicional para mantener la calidad.
    \item Simplificación binaria del estadío: El enfoque adoptado distingue únicamente entre estadío inicial y avanzado, asumiendo dos categorías discretas de progresión tumoral. En la práctica clínica, el cáncer de mama abarca un espectro continuo de estadíos intermedios, por tanto, reducir la clasificación a solo dos niveles puede omitir matices importantes. Esta simplificación binaria limita el modelo a escenarios extremos de la enfermedad y podría no captar transiciones graduales o casos limítrofes entre estadíos.
    \item Alcance de la validación experta: La validación con radiólogos, si bien valiosa, se basó en un número acotado de expertos y de imágenes evaluadas. Esto supone un posible sesgo estadístico y no refleja toda la variabilidad de opiniones que podría obtenerse de una muestra más amplia de especialistas.
\end{itemize}

\subsection{Limitaciones técnicas del proyecto}

Es importante delinear las restricciones técnicas bajo las cuales se llevó a cabo esta investigación, dado que han moldeado en gran medida las decisiones metodológicas. El trabajo se desarrolló utilizando principalmente un equipo de cómputo personal de características modestas: una computadora portátil con procesador AMD Ryzen 7, 16 GB de memoria RAM y una unidad de almacenamiento sólido (SSD) de 512 GB. Estos recursos resultan limitados para las exigencias del entrenamiento de modelos de aprendizaje profundo, especialmente tratándose de redes generativas adversariales que suelen requerir gran poder de cómputo y memoria para procesar lotes de imágenes de alta resolución.

En vista de lo anterior, se contó con el apoyo complementario de un servidor remoto equipado con una unidad de procesamiento gráfico (GPU) dedicada, en el cual se llevaron a cabo los experimentos de entrenamiento más intensivos. En particular, motivaron la elección de una red GAN compacta con un número reducido de parámetros y optimizaciones que disminuyen su demanda de cómputo, permitiendo entrenarla en un entorno de hardware modesto sin incurrir en tiempos de entrenamiento prohibitivamente largos. Del mismo modo, el tamaño de las imágenes y del lote (\emph{batch size}) durante el entrenamiento tuvo que ajustarse para acomodarse a los recursos computaciones disponibles, equilibrando la resolución de las mastografías con la factibilidad de procesamiento.

Otra consecuencia de las restricciones de hardware fue la necesidad de simplificar ciertos experimentos y limitar el alcance de las pruebas. Por ejemplo, no fue posible explorar configuraciones muy extensas de hiperparámetros ni realizar entrenamientos exhaustivos con todas las variaciones posibles de la arquitectura, debido al tiempo de cómputo que ello implicaría. En suma, las condiciones técnicas definieron un entorno realista en el cual la investigación buscó desenvolverse, emulando las circunstancias que muchos centros académicos o clínicos con recursos limitados enfrentan al intentar aplicar técnicas avanzadas de inteligencia artificial. Lejos de verse únicamente como un obstáculo, estas limitaciones también dieron sentido al objetivo de compacidad del modelo y de lograr resultados relevantes con una solución eficiente y accesible.

\section{Estructura de la tesis}

La presente tesis se organiza en siete capítulos, cuyos contenidos se describen a continuación:

\begin{itemize}
    \item En el capítulo 1 ``Introducción'', se expone el contexto general y específico del cáncer de mama, se plantea el problema de investigación, la justificación motivación del estudio, el objetivo general y objetivos específicos, la hipótesis, la solución propuesta, las limitaciones técnicas enfrentadas y una visión general de la estructura del documento.

    \item El capítulo 2 ``Marco teórico'' reúne los fundamentos teóricos necesarios para comprender el trabajo. En este capítulo se abordan conceptos  enfocados a las redes generativas adversariales y su aplicación en imágenes de mamografías. Este marco sienta las bases conceptuales sobre las cuales se desarrolla la propuesta.

    \item En el capítulo 3 ``Estado del arte'', se realiza una revisión de trabajos previos y avances recientes relacionados con el pronóstico de lesiones en imágenes médicas y, específicamente, en mastografías. Se discuten investigaciones enfocadas en la predicción de riesgo de cáncer a partir de imágenes, en simulación de progresión de tumores con métodos computacionales y en el uso de GAN u otras técnicas generativas para síntesis de imágenes médicas. Esta sección permite identificar las contribuciones previas, las metodologías empleadas y los vacíos de conocimiento que la presente tesis busca abordar, posicionando claramente la originalidad y relevancia del estudio.

    \item El capítulo 4 ``Metodología'' describe detalladamente el enfoque propuesto para resolver el problema. Se incluyen la caracterización del conjunto de datos utilizado, la descripción de la arquitectura de la red GAN condicional compacta diseñada, los detalles del procedimiento de entrenamiento, así como consideraciones prácticas implementadas para sortear las limitaciones técnicas. También se explica en este capítulo cómo se planearon los experimentos para evaluar el modelo, incluyendo qué comparaciones o análisis se realizarían para juzgar su desempeño.

    \item En el capítulo 5 ``Resultados experimentales'', se presentan los hallazgos obtenidos al aplicar la metodología. Se muestran los resultados cuantitativos (valores de métricas de similitud y error entre las imágenes pronosticadas por la GAN y las imágenes reales de seguimiento) y cualitativos (ejemplos visuales de mastografías reales contra sintéticas generadas). Asimismo, se incluyen ilustraciones de casos de estudio relevantes, por ejemplo, casos en los que el modelo predice adecuadamente un crecimiento significativo del tumor, o casos en que no logra capturar ciertos cambios. Estos resultados se analizan y se contrastan con las expectativas: se discute en qué medida el modelo cumplió con pronosticar la apariencia de los tumores, cuáles fueron sus aciertos, y cuáles sus limitaciones o errores.

    \item El capítulo 6 ``Conclusiones'' sintetiza los aportes más importantes de la investigación. En este apartado se responden las preguntas planteadas en el problema de investigación, se evalúa el grado en el que se cumplieron los objetivos propuestos y se destacan las contribuciones originales del trabajo. Las conclusiones establecen la relevancia de los resultados obtenidos en el contexto más amplio de la detección y seguimiento del cáncer de mama, y reflexionan sobre las implicaciones de incorporar herramientas basadas en inteligencia artificial para el pronóstico de imágenes en la práctica médica. También se ofrecen sugerencias y perspectivas a futuro derivadas de la experiencia obtenida. Por un lado, se proponen recomendaciones prácticas dirigidas a instituciones de salud o investigadores que deseen implementar o mejorar sistemas similares al desarrollado, señalando consideraciones de hardware, tamaño de muestra, o integración con flujos clínicos. Por otro lado, se plantean posibles líneas de investigación futura que quedaron fuera del alcance de esta tesis en busca que el trabajo trascienda más allá de este documento y sirva de base para continuos avances en la materia.
\end{itemize}